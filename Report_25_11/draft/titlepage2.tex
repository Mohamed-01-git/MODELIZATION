\thispagestyle{empty} % Supprime les en-têtes et pieds de page pour cette page

\clearpage
\begin{titlepage}
  \begin{center}
    % Logos at the top
    \includegraphics[width=0.15\textwidth]{DGM.jpg} \hfill
    \includegraphics[width=0.25\textwidth]{ehtp.jpg} \\[2cm]

    
    % Colored lines above and below the main title
    \noindent\rule{\textwidth}{1mm}\\[0.5cm]
    {\LARGE \textbf{\textcolor{blue}{Evaluation of Climate Models For Seasonal Forecasting in the MENA Region\\ \vspace{0.1cm} 
   }}}
    \noindent\rule{\textwidth}{1mm}\\[3cm]
    \vfill
    % Author and supervisor
   \begin{tabbing}
    \hspace{6cm} \= \kill 
    \textbf{\large Prepared by:} \> \textbf{Berrahmouch Nohayla and Mohamed El-Badri} \\[0.2cm]
    \> Hassania School of Public Works , Casablanca, Morocco \\[1cm]
    
    \textbf{\large Supervised by:} \> \textbf{Mrs. Wafae Badi and Mr. Nicholas Savage} \\[0.2cm]
    \> Direction Générale de la Météorologie, Morocco (Wafae Badi) \\[0.2cm]
    \> Met Office, Exeter, UK (Nicholas Savage) \\
\end{tabbing}

    \vfill

    % Date
    {\large 2024 - 2025} \\[1cm]
  \end{center}
\end{titlepage}
\tableofcontents
\newpage
\chapter*{Acknowledgments}
\addcontentsline{toc}{chapter}{Acknowledgments}
\vskip-3.3em  


We would like to express our deepest gratitude to \textbf{Mrs. Wafae Badi} for her unwavering support and invaluable guidance throughout this project. Her insightful counsel and constant encouragement enabled us to overcome various challenges and maintain our focus. Her dedication to fostering progress, coupled with her constructive feedback and open-minded approach, were instrumental in shaping the direction of this work. Her mentorship has truly been a cornerstone of our journey, and we are profoundly grateful for her invaluable contributions.  
\\

Special thanks go to \textbf{Mr. Nicholas Savage} and his exceptional team at the UK Met Office. Their generosity in sharing their expertise and resources provided us with unparalleled opportunities to broaden our understanding of climate modeling. The engaging discussions and valuable insights shared by Mr. Savage and his team not only enriched this project but also fueled our motivation to explore innovative avenues. Their commitment to advancing climate science inspired us to aim higher and achieve more.
\\

We are also immensely grateful to \textbf{Mr. Bari}, whose dedicated supervision, thoughtful suggestions, and constructive critiques significantly enhanced the quality of this work. His ability to balance critical feedback with motivating encouragement made a remarkable difference, guiding us through challenging moments and ensuring steady progress. 
\\

In addition, we would like to acknowledge the support received through the \textbf{WISER MENA project}. \textbf{Nicholas Savage’s time was funded via the WISER MENA project.} The Weather and Climate Information Services (WISER) Programme is funded with UK International Development from the UK government and led by the Met Office in the UK. This work has been partially supported by UK International Development from the UK government; however, the views expressed do not necessarily reflect the UK government’s official policies.  
\\

Lastly, we extend our heartfelt appreciation to all those who, directly or indirectly, contributed to this project. Your cooperation, guidance, and belief in our work have made this journey a fulfilling and enlightening experience. While this project is a testament to hard work and collaboration, it is also a reflection of the collective effort and support of everyone who believed in its success. To you, we owe our sincere thanks.

\chapter*{Preface}
\addcontentsline{toc}{chapter}{Preface}
\vskip-3.3em    
The MENA seasonal forecasting models have undergone both probabilistic and deterministic evaluations. This research study is regarded as the pioneering work and the first of its kind in this area which helps in situational context improvement in seasonal forecasting models. Given the alarming rate of increase in the impacts caused by extreme climatic events including severe droughts, and extreme heat and other climate sensitive issues in the MENA region, this work is a key contribution towards alleviating these issues=
\\

Due to climatic extremes in the MENA region, agriculture, human livelihood, and natural resources are heavily affected. Consequently, it has become almost necessary to have forecasts of seasons that are credible so as to characterize the impacts, or to enhance preparedness. Although seasonal forecasting models have been widely researched and practiced in many parts of the world, their use in MENA countries’ local level remains scarce. This gap is resolved in this study, providing new knowledge and tools for climate scientists working in the region.\\

In this work, we intend to broaden the knowledge fabric of climate change science by focusing on the climate change and variability vulnerability of the MENA region. The results obtained not only improve the comprehension of the dynamics of the local climate, but also lays a framework for specific approach to be employed for adaptation strategies.\\

We are immensely grateful to every individual or organization who has helped support this project and guided us through uncharted territory in the spectrum of MENA climate predictions.\\
\newpage
\chapter*{Overview and Rationale of the Study}
\addcontentsline{toc}{chapter}{Overview and Rationale of the Study}
The last couple of decades have witnessed a surge in demand for seasonal climate forecasting. Global advancements in space science and technology have lead to the better anticipation of climate seasons up to a through range of 3-12 months. This is crucial for effective planning in major industries like agriculture or energy management, among others. These advancements breed an increased dependence on seasonal forecasting and in turn create a higher demand for accurate forecasting mechanisms. Therefore two central methodologies have witnessed prominence – deterministic and probabilistic methods. A hindsight understanding of these mechanisms is imperative, as they are useful for evaluating and understanding the shortcomings and effectiveness of different models employed in forecasting seasonal amps.\\

Probabilistic forecasts take one step forward, do not try to predict an ideal scenario and present different potential outcomes, each with a defined probability. Efforts, though different, instruct towards the same ends; meeting a specific operational/strategic need. Lorenz’s butterfly effect presents the case for one such endeavor- it shows how a non-linear system’s response can drastically alter depending on the initial conditions. Such chaos is especially present in weather and climate systems where even the slightest details can have large ramifications over longer periods.\\

The study on the other hand tries to develop such relationships that integrate conceptual developments in seasonal forecasting efforts with applicable methods.