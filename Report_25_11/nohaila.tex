
\section*{Acknowledgments}
Most importantly, we would like to give special thanks and deep appreciation to \textbf{Mrs. Waafae Badi}. Her endless support and sagacious counsel encouraged us to withstand various difficulties under the project. The feedback, knowledge, and open manner with which she was willing to work to help us face some of the barriers encountered are praised with satisfaction. Because of her guidance pushing us to work harder, our journey would not have been as productive.\\

Nicholas Savage’s time was funded via the WISER MENA project. The Weather and Climate Information Services (WISER) Programme is funded with UK International Development from the UK government and led by the Met Office in the UK. This work has been partially supported by UK International Development from the UK government; however, the views expressed do not necessarily reflect the UK government’s official policies. \\ 


Many thanks to \textbf{Mr. Nicholas Savage} and his great team at the UK Met Office; it was because of their immense generosity and knowledge with considerable resource selection that our work was truly able to traverse actual lines. New doors opened for us that day; discussions and interactions with them broadened our horizons for climate modeling and energized the project itself. Their dedication and commitment toward climate science raised the spirits and empowered the aspiration to reach high.\\

Finally, we wish to extend our considerable gratitude to \textbf{Mr. Bari}, whose position was the supervisor providing guidance and assessing progress on the project. His intelligent suggestions, moderate-though-motivating feedback, and consistent willingness to provide practical assistance during our time of troubles enhanced the preparation of this work. We really appreciated his outstanding, constant support to promote progress and guarantee quality throughout the lifetime of this project.



In this context, we would like to extend our profound gratitude to all those whose fingers contributed to this project. Their cooperation and guidance have ensured that some aspects of the path were easier and enriched the process. But above all, this project is not the work, so it concludes, of the supposedly successful; it is, instead, a conundrum expedition of all those who believed in us. 
\newpage

\section*{Preface}
The MENA seasonal forecasting models have undergone both probabilistic and deterministic evaluations. This research study is regarded as the pioneering work and the first of its kind in this area which helps in situational context improvement in seasonal forecasting models. Given the alarming rate of increase in the impacts caused by extreme climatic events including severe droughts, and extreme heat and other climate sensitive issues in the MENA region, this work is a key contribution towards alleviating these issues=
\\

Due to climatic extremes in the MENA region, agriculture, human livelihood, and natural resources are heavily affected. Consequently, it has become almost necessary to have forecasts of seasons that are credible so as to characterize the impacts, or to enhance preparedness. Although seasonal forecasting models have been widely researched and practiced in many parts of the world, their use in MENA countries’ local level remains scarce. This gap is resolved in this study, providing new knowledge and tools for climate scientists working in the region.\\

In this work, we intend to broaden the knowledge fabric of climate change science by focusing on the climate change and variability vulnerability of the MENA region. The results obtained not only improve the comprehension of the dynamics of the local climate, but also lays a framework for specific approach to be employed for adaptation strategies.\\

We are immensely grateful to every individual or organization who has helped support this project and guided us through uncharted territory in the spectrum of MENA climate predictions.\\
\newpage
\section*{Overview and Rationale of the Study}
The last couple of decades have witnessed a surge in demand for seasonal climate forecasting. Global advancements in space science and technology have lead to the better anticipation of climate seasons up to a thorugh range of 3-12 months. This is crucial for effective planning in major industries like agriculture or energy management, amongst others. These advancements breed an increased dependence on seasonal forecasting and in turn create a higher demand for accurate forecasting mechanisms. Therefore two central methodologies have witnessed prominence – deterministic and probabilistic methods. A hindsight understanding of these mechanisms is imperative, as they are useful for evaluating and understanding the shortcomings and effectiveness of different models employed in forecasting seasonal amps.\\

Probabilistic forecasts take one step forward, do not try to predict an ideal scenario and present different potential outcomes, each with a defined probability. Efforts, though different, instruct towards the same ends; meeting a specific operational/strategic need. Lorenz’s butterfly effect presents the case for one such endeavor- it shows how a non-linear system’s response can drastically alter depending on the initial conditions. Such chaos is especially present in weather and climate systems where even the slightest details can have large ramifications over longer periods.\\

The study on the other hand tries to develop such relationships that integrate conceptual developments in seasonal forecasting efforts with applicable methods.
\newpage

\section{Introduction}
\subsection{Context}
\subsubsection{Overview of Climate Modeling and Seasonal Forecasting}
Climate modeling is the process of using mathematical representations of the Earth’s atmosphere, oceans, land surface, and ice systems to simulate and predict climate dynamics. These models are based on fundamental physical principles, such as the conservation of mass, energy, and momentum, and are implemented through numerical methods that solve complex equations governing the interactions between these systems.\footnote{McGuffie, K. and Henderson-Sellers, A., 2014. A Climate Modelling Primer. \url{https://doi.org/10.1002/9781118687853}} Climate models range from global circulation models (GCMs), which simulate large-scale atmospheric and oceanic processes, to regional climate models (RCMs), which provide localized projections by incorporating finer-scale topographic and land-use details.\footnote{Flato et al., 2013. Evaluation of Climate Models. IPCC AR5 Chapter 9. \url{https://www.ipcc.ch/report/ar5/wg1/chapter-9-evaluation-of-climate-models/}} Seasonal forecasting, a subset of climate modeling, refers to the prediction of climate conditions, such as temperature and precipitation, over a period of one to six months. These forecasts rely on initial conditions (e.g., sea surface temperatures, soil moisture) and slowly varying components of the climate system, such as oceanic or atmospheric anomalies like the El Niño-Southern Oscillation (ENSO).\footnote{Doblas-Reyes, F. J., García-Serrano, J., Lienert, F., Biescas, A. P., \& Rodrigues, L. R., 2013. Seasonal climate predictability and forecasting: Status and prospects. \url{https://doi.org/10.1038/ngeo1714}} The basic principle behind seasonal forecasting is to leverage these slowly varying components, which have a predictable influence on regional weather patterns, using ensemble simulations to quantify uncertainties and provide probabilistic predictions.\footnote{Palmer, T. N., \& Anderson, D. L., 1994. The prospects for seasonal forecasting—a review paper. \url{https://doi.org/10.1256/smsqj.50402}}  

Seasonal forecasts play a crucial role in decision-making and planning across various sectors, including agriculture, water management, and climate risk mitigation. These forecasts provide early warnings of high-impact climate scenarios, enabling proactive decisions that result in financial savings, risk reduction, and optimized resource use. For instance, in agriculture, they assist farmers in selecting appropriate crops and determining optimal planting times based on anticipated water availability, thereby mitigating risks associated with droughts or excessive rainfall.\footnote{Werner, M. and Linés, C., 2024. Seasonal forecasts to support cropping decisions. \url{https://doi.org/10.5194/egusphere-egu24-13436}} Seasonal forecasts also support pre-harvest strategies, such as hedging decisions, which help shield farmers from price volatility, although their adoption is often hindered by perceptions of inaccuracy and complexity.\footnote{Hunt et al., 2020. Seasonal Forecast Based Preharvest Hedging. \url{https://doi.org/10.22004/AG.ECON.309761}} In water management, seasonal forecasts are vital for mitigating drought impacts, particularly in semi-arid regions, by enabling improved reservoir operations and efficient water allocation to reduce losses.\footnote{Portele et al., 2021. Seasonal forecasts offer economic benefits for hydrological decision-making. \url{https://doi.org/10.1038/s41598-021-89564-y}} Additionally, these forecasts, when linked to hydrological models, improve predictions of water balance and inform critical decisions regarding water storage and distribution, despite occasional discrepancies between predicted and desired variables.\footnote{MacLeod et al., 2023. Translating seasonal climate forecasts into water balance forecasts. \url{https://doi.org/10.1371/journal.pclm.0000138}} Seasonal forecasts are increasingly applied in climate risk management, where they help predict extreme weather events, providing decision-makers with tools to minimize societal and economic damages.\footnote{Castino et al., 2023. Towards seasonal prediction of extreme temperature indices. \url{https://doi.org/10.5194/ems2023-590}} For example, accurate predictions of heatwaves or floods allow authorities to implement adaptive measures, reducing infrastructure damage and safeguarding public health. In economic sectors such as energy and water management, tailored seasonal forecasts enhance decision-making efficiency by aligning forecasts with user needs, thereby optimizing outcomes.\footnote{Goodess et al., 2022. The Value-Add of Tailored Seasonal Forecast Information. \url{https://doi.org/10.3390/cli10100152}} Despite their significant potential, the effectiveness of seasonal forecasts depends on their accuracy, relevance to user needs, and ease of use. Improved communication, stakeholder training, and efforts to bridge the gap between forecast complexity and user understanding are essential to maximize their utility.



\subsubsection{Importance of Seasonal Climate Forecasts in MENA}
Seasonal climate forecasts are critically important across the MENA region, where high temperatures, low water availability, and vulnerability to climate variability create substantial challenges for sustainable development. Forecasts provide early warnings of droughts, heatwaves, and other extreme weather events, enabling decision-makers to implement proactive measures to mitigate impacts on water resources, agriculture, and infrastructure.\footnote{Dunn et al., 2020. The changing climate of MENA. \url{https://pubs.giss.nasa.gov/abs/gu00200u.html}} In agriculture, these forecasts help farmers optimize crop selection and planting schedules, reducing the risks of crop failure in this water-scarce region.\footnote{Werner, M., and Linés, C., 2024. Seasonal forecasts to support cropping decisions. \url{https://doi.org/10.5194/egusphere-egu24-13436}} In the water sector, seasonal forecasts guide reservoir management by predicting rainfall variability, improving water storage strategies, and ensuring more equitable water distribution.\footnote{Portele et al., 2021. Seasonal forecasts for hydrological decision-making. \url{https://doi.org/10.1038/s41598-021-89564-y}} With increasing climate risks, these forecasts also support disaster risk management by allowing governments to prepare for extreme events, such as heatwaves and floods, which are becoming more frequent in the region due to climate change.\footnote{Castino et al., 2023. Towards seasonal prediction of extreme temperature indices. \url{https://doi.org/10.5194/ems2023-590}} Moreover, the economic benefits of using seasonal forecasts are significant. By enabling energy companies to anticipate peak demand periods driven by heatwaves, and by helping municipalities optimize water usage during droughts, these forecasts provide cost savings and efficiency gains.\footnote{Goodess et al., 2022. Value-Add of tailored seasonal forecast information. \url{https://doi.org/10.3390/cli10100152}} However, challenges persist in ensuring the accuracy and usability of these forecasts. The arid and semi-arid nature of much of the MENA region, coupled with complex interactions between regional climate drivers, makes it difficult to provide highly localized forecasts.\footnote{Latif et al., 2011. ENSO predictability and regional climate impacts. \url{https://doi.org/10.1175/2010JCLI3405.1}} Addressing these challenges through improved modeling techniques and stakeholder engagement will be critical to maximizing the value of seasonal forecasts in the MENA region, ensuring better preparedness and resilience against a changing climate.


\subsection{Objectives of the Work}
The primary objective of this work is to evaluate the effectiveness of climate models, focusing specifically on their performance in predicting key climate variables such as temperature, precipitation. This evaluation incorporates both deterministic and probabilistic approaches to identify the most skillful models and their suitability for practical applications.

\subsubsection{Specific aims  of evaluating deterministic and probabilistic models.
}
The evaluation of deterministic and probabilistic models is essential for understanding their unique strengths, limitations, and potential applications in diverse fields. Deterministic models, which generate a single, precise outcome based on initial conditions, are widely used when exactness and reproducibility are critical, such as in engineering and physical simulations.\footnote{McGuffie, K., and Henderson-Sellers, A., 2014. *A Climate Modelling Primer*. Wiley. \url{https://doi.org/10.1002/9781118687870}} Their evaluation focuses on assessing accuracy and reliability under specific conditions, providing clarity in cause-and-effect relationships. In contrast, probabilistic models incorporate uncertainty by assigning probabilities to various potential outcomes, enabling the representation of real-world complexities and variability.\footnote{Palmer, T., and Hagedorn, R., 2006. *Predictability of Weather and Climate*. Cambridge University Press. \url{https://doi.org/10.1017/CBO9780511617652}} These models are particularly beneficial for strategic planning and risk management, where understanding a range of possible scenarios is crucial. The evaluation of both types of models includes conducting sensitivity analyses to determine how changes in input variables affect outcomes, which helps in identifying key drivers of uncertainty and improving model performance.\footnote{Seneviratne, S.I., et al., 2021. *Metrics for climate model evaluation: A review*. Nature Communications. \url{https://doi.org/10.1038/s43247-021-00094-x}} Additionally, risk assessment is a vital component, with deterministic approaches offering straightforward estimations for defined scenarios, while probabilistic approaches address uncertainties by simulating a spectrum of possible outcomes.\footnote{PreventionWeb, 2021. *Deterministic and Probabilistic Risk*. \url{https://www.preventionweb.net/understanding-disaster-risk/key-concepts/deterministic-probabilistic-risk}} These evaluations also aim to support decision-making processes by identifying which type of model is more appropriate for specific contexts—deterministic models for precise predictions and probabilistic models for flexible planning under uncertainty.\footnote{Goodess, C.M., et al., 2022. *The Value-Add of Tailored Seasonal Forecast Information for Industry Decision Making*. Climate. \url{https://doi.org/10.3390/cli10100152}} Finally, probabilistic models are often recognized for their adaptability in dynamic environments, as they can incorporate new data and adjust probability distributions to reflect evolving conditions, making them indispensable for complex systems where deterministic models may fall short.\footnote{Latif, M., and Keenlyside, N., 2011. *El Niño/Southern Oscillation Predictability*. Journal of Climate. \url{https://doi.org/10.1175/2010JCLI3405.1}} Together, the evaluation of deterministic and probabilistic models provides invaluable insights into their suitability for addressing specific challenges, supporting informed decision-making, and advancing model development.

\subsubsection{Description of Content}

This report is designed to provide a comprehensive analysis of climate model evaluation, focusing on both deterministic and probabilistic approaches. The structure of the report follows a logical progression, starting with an introduction to the fundamental concepts behind climate models. The first section lays the groundwork for understanding the key differences between deterministic and probabilistic models, describing how each approach is used to simulate climate systems and predict future outcomes. The methodology chapter follows, detailing the specific techniques employed to assess the models. This includes the use of both deterministic and probabilistic metrics such as Root Mean Square Error (RMSE), Anomaly Correlation Coefficient (ACC), and Brier Score, which are critical for evaluating the models' accuracy and performance in predicting climate variables like temperature and precipitation.

Next, the report moves on to the results and analysis, where the performance of the selected models is presented and compared. This section highlights the models' strengths and weaknesses, providing insight into how well they predict climate patterns across various geographical regions and time periods. Special attention is given to the models' skill in forecasting extreme weather events, which are particularly relevant to sectors like agriculture, water resource management, and disaster risk reduction.

The final section of the report provides conclusions and recommendations based on the analysis. This chapter synthesizes the findings, offering practical suggestions for improving the accuracy, usability, and application of climate forecasts. Recommendations also address how future developments in climate modeling can better meet the needs of decision-makers and stakeholders. The report as a whole seeks to contribute valuable insights into the ongoing development of climate prediction systems, aiming to enhance their effectiveness in real-world applications.






